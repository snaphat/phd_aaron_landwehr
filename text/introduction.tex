\label{chap:introduction}

Infrared Scene Projection Systems (IRSPs) are emerging as a novel technology for the testing and development of Infrared (IR) based sensor technology and real-time IR simulations. They provide a compelling alternative to the older entrenched technology of resistor-array based IR scene projector systems~\cite{PritchardEtAl1998,WilliamsEtAl2005} due to various improvements over the competing technology. These improvements include but are not limited to better maximum apparent temperature\footnote{The temperature a black body giving the same radiance would be.} (above 1400 Kelvin), better dynamic range, higher pixel density (24 micrometers and lower), substantially faster emission in the target spectrums with optical rise-times in the nanoseconds. Additionally, they are relatively difficult to damage thermally and have the potential to provide emission in multiple spectrums with \emph{multi-color pixel} designs.

Current fixed frame rate display technology, such as, DVI~\cite{DDWG1999}, HDMI~\cite{HDMIForum2018}, and DisplayPort~\cite{BhowmikEtAl2012} is commonly utilized for high-speed IRSPs. It provides a standardized method to transmit digital scene data which is then translated into analog signaling for display on IR arrays. However, within IRLED IRSP systems which are inherently fast and primarily limited by the driving electronics, it has become limiting for high-speed display~\cite{LaVeignePrewarski2013}. This technology, designed for relatively low-speed operation, incorporates a number of design decisions that limit the ability for it to meet the increasing requirements of larger resolutions and faster frame rates needed within IR display systems. Firstly, it requires custom designed synchronization solutions and hardware when utilized within environments where multiple components need to be synchronized. This is because it is not designed to handle system level synchronization. For example, a display wall may utilize Quadro Sync Cards~\cite{NVIDIA2020_2} to provide synchronization across multiple monitors; however, this only guarantees a coarse-grain synchronization between displays. Secondly, the fixed frame rate nature of the technology imposes a static requirement on frame rate across all displayed frames. This unnecessarily increases bandwidth demands by requiring the same amount of data be sent for all frames regardless of what data changes. As a result, maximum frame rate unnecessarily becomes a function of limited hardware bandwidth and image resolution. This relationship between frame size and bandwidth is discussed in more detail in Chapter~\ref{chap:problem_formulation}.

This dissertation proposes an alternative to conventional display technology, a packetized display protocol (PDP) architecture capable of providing a synergy with IRSP technology to bridge the performance gaps within existing display solutions for current IR display systems. Sensor technology can operate in ranges of above one kilohertz which represents an order of magnitude difference to the current target speeds of fixed-rate display technology. This PDP architecture eschews with many assumptions found in conventional display protocol technology. In doing so, it provides scalability, reduces bandwidth requirements, increases performance, eases synchronization burden as well as provides a desirable set of features for current and future IRLED Scene Projection systems. These features include dynamic sub-window (intra-frame) refresh rates, dynamic bandwidth utilization, and dynamic inter-frame refresh rates. Furthermore, this dissertation contributes a protocol specification and implementation on real hardware, coupled with a demonstration of this type of technology within real IRSP systems which shows that with the proper set of control and features, high-speed operation can be achieved even with limited physical bandwidth.

The protocol architecture draws inspiration from the video processing field, where encoding schemes for video streaming represent a body of research that attempts to tackle a similar but more limited challenge~\cite{BakarEtAl2017}. Some of these encoding schemes attempt to provide a variable frame rate for segments of the incoming stream through differencing algorithms, but also rely on compression~\cite{CastilloEtAl2012} which reduces quality and introduces artifacts. In contrast, the case of IRSPs requires lossless quality; and thus, lossy protocols cannot be utilized for this purpose. Instead, the proposed protocol architecture seeks to craft a lossless solution for the IRLED projector field that incorporates similar variable frame rate features to reduce bandwidth consumption as well as allow bandwidth to be used more intelligently. More specifically, it is envisioned that available bandwidth will be apportioned to regions of a scene that necessarily need to be updated frequently. In IR scenes, this generally includes regions that transition from dark to light or light to dark quickly, as well as higher temperature regions. Regions which do not change temperature quickly, generally do not need to be updated as often due to the LED driving circuits holding capacitance for milliseconds at a time\footnote{The general time of discharge depends on the design of the LEDs and amount of charge currently held within. However, test setups have measured \textgreater1 millisecond.}.

The contributions of this dissertation are as follows: firstly, it provides the architecture of a physical layer agnostic packetized display protocol with the following features: (1) intelligent dynamic per-frame bandwidth utilization, (2) fine-grained control over frame transmission and synchronization, (3) dynamically changing intra-frame rates, and (4) a realized implementation of the protocol for use on array emitter technology. Within it discusses relevant details of the initial design, methodology, and implementation of the said protocol. Secondly, it provides a sufficiently abstract machine model to indicate a path to utilize the protocol within current and future systems. Thirdly, it demonstrates the use of the protocol within real IRSP systems as well as provides the current results and a comparison with fixed-rate technology. Fourthly, it discusses various use cases for the technology to provide the reader with a more complete understanding of where this technology could be utilized in future systems.

The rest of this dissertation is divided into the following sections: (1) background; which discusses the various aspects of current IRSP systems that are relevant to understanding the IRLED scene projector history and projection process, (2) problem formulation; which examines the problem of high-speed projection in detail, (3) system overview; which discusses the supporting electronics and communication flow within IRLED projector systems, (4) array write process; which discusses how IR arrays and the associated data ordering, (5) display protocols; which discusses conventional display protocols and their use within IRSP technology, (6) packetized display protocol; which discusses the design methodology for the PDP, packet details, overhead, and performance, as well as provides a comparison to conventional display protocols; (7) machine model; which discusses the use of the PDP in general systems; (8) implementation; which discusses an implementation of the PDP on an FPGA system, (9) experimental results; which provides details on the implementation and testing process as well as a discussion on performance; and (10) conclusion; which discusses the future of PDP and potential avenues of further research.
