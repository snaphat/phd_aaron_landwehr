\label{chap:introduction}

Infrared Scene Projection Systems (IRSPs) are emerging as a novel technology for the testing and development of Infrared (IR) based sensor technology and real-time IR simulations. They provide a compelling alternative to the older entrenched technology of resistor-array based IR scene projector systems\cite{pritchard1998design,williams2005history} due to various improvements over the competing technology. These improvements include but are not limited to better maximum apparent temperature (above 1400 Kelvin), better dynamic range, higher pixel density (24 micrometers and lower), substantially faster emission in the target spectrums with optical rise-times in the nanoseconds. Additionally, they are relatively difficult to damage thermally and have the potential to provide emission in multiple spectrums with \emph{multi-color pixel} designs.

Current fixed frame rate display technology, such as, DVI\cite{DDWG1999}, HDMI\cite{HDMIForum2018}, and DisplayPort\cite{BhowmikEtAl2012} are commonly utilized technologies within IRSP and resistor array systems. They provide a standardized method to transmit digital scene data which is then translated into analog signaling for display on IR arrays. However, within IRLED systems which are inherently fast and primarily limited by electronics, these technologies have become limiting for high-speed IR display\cite{EjzakEtAl2016,LaVeignePrewarski2013}.

These display technologies, designed for relatively low-speeds (generally 60Hz until recently) incorporate a number of design decisions that limit the ability to utilize them effectively with IR emitter technology. Firstly, these technologies require custom designed synchronization solutions and hardware when utilized with multiple sources in order to ensure correct synchronization because they are not designed to handle synchronization across multiple sources. For example, display walls may utilize Quadro Sync Cards\cite{NVIDIAQuadroSync} to provide synchronization across multiple monitors; however, this only guarantees  a coarse-grain synchronization within one frame of latency between displays. Secondly, the fixed frame rate nature of technology imposes a static requirement on frame rate across all displayed frames increasing bandwidth requirements by requiring the same amount of data be sent for all frames regardless of what data changes. This necessarily means that maximum frame rate operation is limited by the resolution size of imagery due to bandwidth limitations across physical links. This relationship between frame size and bandwidth is discussed in more detail in Chapter~\ref{chap:problem_formulation}.

This dissertation proposes an alternative to traditional display technology, a packetized display protocol (PDP) architecture capable of providing a synergy with the the benefits of IRSP technology in order to bridge the performance gap of ever the increasing speed requirements of high-speed projector systems. Sensor technology can operate in ranges of above one kilohertz which represents an order of magnitude difference to the current target speeds of fixed-rate display technology. This PDP architecture eschews with many of the assumptions found within traditional display technology in order to provide scalability, reduce bandwidth requirements, increase performance, ease synchronization burden, as well as; provide a desirable set of features, such as, dynamic sub-window frame rates not found within current fixed-rate technology. This architecture demonstrates that with the proper set of control and features, high-speed operation can be achieved even with limited physical bandwidth.
%FIXME: do kilohertz sensors exist?

 The protocol architecture draws inspiration from the video processing field, where encoding schemes for video streaming represent a body of research that attempts to tackle a similar but more limited challenge\cite{BakarEtAl2017}. Some of these encoding schemes attempt to provide a variable frame rate for segments of the incoming stream through differencing algorithms, but also rely on compression\cite{CastilloEtAl2012} which reduces quality and introduce artifacts. In contrast, the case of IRSPs requires lossless quality; and thus, lossy protocols cannot be utilize for this purpose. Instead, the proposed protocol architecture seek to craft a lossless solution for the IRLED projector field that incorporates similar variable frame rate features in order to reduce bandwidth consumption, as well as, allow bandwidth to used more intelligently. More specifically, it is envisioned that avaliable bandwidth will be apportioned to regions of a scene that necessarily need to be updated frequently. In IR scenes, this generally includes regions that transition from dark to light or light to dark quickly, as well as higher temperature regions. Regions which do not change temperature quickly, generally do not need to be updated as often due to the LED driving circuits holding capacitance for milliseconds at a time\footnote{The general time of discharge depends on the design of the LEDs and amount of charge currently held within. However, test setups have measured \textgreater1 millisecond.}.
%FIXME: How long to LEDs actually hold capacitance, and is there a citation I can use?

The contributions of this dissertation are as follows: firstly, it provides the architecture of a physical layer agnostic packetized display protocol with the following features (1) intelligent dynamic per-frame bandwidth utilization, (2) fine-grained control over frame transmission and synchronization, (3) dynamically changing intra-frame rates, and (4) a realized implementation of the protocol for use on array emitter technology. Within it discusses relevant details of the initial design, methodology, and implementation of the said protocol. Secondly, it provides a sufficiently abstract machine model to indicate a path to utilize the protocol within current and future systems. Thirdly, it demonstrates the use of the protocol within real IRSP systems, as well as, provide the current results and a comparison with fixed-rate technology. Fourthly, it discusses various use-cases for the technology to provide the reader with a more complete understanding of where this technology could be utilized in future systems.

%FIXME: ordering of this
This dissertation is divided into the following sections: background; which discusses the various aspects of current IRSP systems that are relevant to understanding PDP, problem formulation; which examines the problem of high-speed projection in detail, packetized display protocol; which discusses the design methodology and the rationale for PDP, machine model; which discusses the overall abstract PDP architecture, use cases; which examines utilizing PDP within real systems, implementation; which discusses an implementation of the protocol on an FPGA system and provides experimental results, related work; which discusses relevant work within the field of high-speed display systems techology, and the conclusion; which discusses the future of PDP and potential avenues of further research.
