\label{chap:display_protocols}
This chapter discusses the details of display protocols. Firstly, it provides a general discussion of how common display protocols work to send pixel data to a display system (e.g. a television). Then, it discusses how these protocols are utilized within IRSP technology.

\section{Conventional Display Protocols}
    \label{sec:conventional_display_protocols}

    Display specifications such as DSI~\cite{HDMIForum2017}, DVI~\cite{DDWG1999}, HDMI~\cite{HDMIForum2018}, and DisplayPort.~\cite{VESA2016} are the backbone of consumer electronic display devices\footnote{Some newer additions such as Variable Refresh Rate (VRR) and the framing of DisplayPort are discussed in Chapter~\ref{chap:pdp_protocol} to allow for a direct comparison with PDP.} They are utilized in a plethora of devices ranging from televisions, monitors, laptops, smart phones, to embedded devices such as point of sale (POS) terminals. Increasingly, they are being utilized in the ever-increasing smart display market for applications such as registration, product menus, smart watches, etc.

    These generally provide a standardized feature-set, or display protocol, that is rooted in classical analog video specifications (e.g. VGA, Composite)~\cite{NI2018} that utilize scan lines~\cite{Neal1998}. Scan lines are used to provide video timing information to synchronize a display to a given refresh rate. Each scan line consists of an active video region followed by a horizontal blanking period. After all active video scan lines are displayed, a vertical synchronization region is used to indicate the end of a frame.

    An overview of this is shown in Figure~\ref{fig:display_protocol_timing_overview}. The region shown in green is the pixel data for the active video region of the display. It is of size $h_a\cdot v_a$ which represents the number of pixels to display, for example, 1920 by 1080 for a HDTV high-definition video mode~\cite{MythTV2015}. The blanking time regions denote pixel data that is sent but not displayed\footnote{Typically data lines are held low during this period, but sometimes they are used for out-of-band communication to send other information such as audio encoding.}. A scan line consists of pixels made up of the $h_a+h_{fp}+h_{sp}+h_{bp}$ regions. These are the horizontal active size, the horizontal front porch, the horizontal sync pulse, and the horizontal back porch, respectively. $v_a$, the vertical active size, indicates the number of scanlines that make up the active region of the display. The vertical blanking period makes up multiple scanlines and consists of $v_{fp}+v_{sp}+v_{bp}$ scanlines. These are the vertical front porch before the vsync pulse, the vertical sync pulse, and the vertical back porch, respectively. Sync pulses are generally active low, meaning that during active display a sync signal is high as shown in the diagram. Note, that this terminology is consistent with the VESA Coordinated Video Timings (CVT) Standard~\cite{VESA2013}.

    Figure~\ref{fig:display_protocol_line_cross} shows a closeup view of signal lines during the active region of display for two scan lines\footnote{The active pixel count is proportionally smaller to blanking regions than in real modes for illustration purposes.}. A data enable signal denoted by $enable$ is high during the active region shown in green. Following this, it goes low for a period of time denoted by the $h_{fp}+h_{sp}+h_{bp}$ regions. The horizontal sync signal goes low only in the region shown in yellow between the front porch and back porches. This process repeats for all scan lines. Once the last active region pixel is drawn, the enable signal will stop going high during the vertical synchronization period.

    %FIXME: Fix discussion of DP not using backwards compatibility HDMI mode?
    %FIXME: Talk about CC in the vertical blanking

    Figure~\ref{fig:display_protocol_full_cross} shows a closeup view of signal lines during the transition into the vertical synchronization period\footnote{The blanking regions are less scanlines than in real modes for illustration purposes.}. The region donated by $v_a$ indicates the end of the video active region of the display which occurs toward the end of a frame. After the active video region, all data has been drawn to a display. The region denoted by $v_{fp}+v_{sp}+v_{bp}$ is the vertical blanking or vsync period during which no active video data is sent; therefore, data enable denoted by $enable$ is always low during this period. Before the vertical sync pulse period denoted by $v_{sp}$ occurs, a vertical front porch period denoted by $v_{fp}$ occurs. After the vertical sync pulse, a vertical back porch region $v_{bp}$ occurs. Following this, the beginning of the next frame begins after the $v_{bp}$ region.

    \begin{figure}[H]
        \centering
        \includegraphics[width=1.0\textwidth]{fig/display_timing_overview.pdf}
        \caption{Display Protocol Timing Overview}
        \label{fig:display_protocol_timing_overview}
    \end{figure}

    \begin{figure}[H]
        \centering
        \includegraphics[width=1.0\textwidth]{fig/display_timing_line_cross.pdf}
        \caption{Display Protocol Horizontal Signal Cross Section Timing}
        \label{fig:display_protocol_line_cross}
    \end{figure}

    \begin{figure}[H]
        \centering
        \includegraphics[width=1.0\textwidth]{fig/display_timing_full_cross.pdf}
        \caption{Display Protocol Full Signal Cross Section Timing}
        \label{fig:display_protocol_full_cross}
    \end{figure}

    Equations~\eqref{eq:l_h}~through~\eqref{eq:f_t} show the relationship between the different regions of a display and the frequency or refresh rate. In Equation~\eqref{eq:l_h}, $l_h$ denotes the scan line size of a display, or total horizontal width, which is made up of the horizontal active and horizontal porch region pixels of a display. In Equation~\eqref{eq:l_v}, $l_v$ denotes the total vertical width of a display, which is made up the vertical active and vertical porch region pixels of a display. In Equation~\eqref{eq:f_f}, each pixel is sent at a rate denoted by $f_p$, the pixel frequency (also called the pixel clock) where the result $f_f$ denotes the frame frequency or frame rate of a display. This is simply the pixel frequency over the total number of pixels (video active and porches) of a display. In Equation~\eqref{eq:p_t}, $p_t$ denotes the time period a single pixel takes to send. In equation~\eqref{eq:f_t}, $f_t$ denotes the time period for an entire frame.


    \begin{equation}
        l_h=h_a+h_{fp}+h_{sp}+h_{bp}
        \label{eq:l_h}
    \end{equation}
    \begin{equation}
        l_v=v_a+v_{fp}+v_{sp}+v_{bp}
        \label{eq:l_v}
    \end{equation}
    \begin{equation}
        f_f={\frac{f_p}{l_h \cdot l_v}}
        \label{eq:f_f}
    \end{equation}
    \begin{equation}
        p_t={\frac{1}{f_p}}
        \label{eq:p_t}
    \end{equation}
    \begin{equation}
        f_t={\frac{1}{f_f}}
        \label{eq:f_t}
    \end{equation}

    To illustrate, let us look at the display modeline generated using the VESA Coordinated Video Timing (CVT) standard shown in Table~\ref{tbl:modeline_example}. This modeline operates a total frame frequency of approximately \mbox{30 Hz}. The pixel clock 79.75, denoted in red, is specified in megahertz. The horizontal pixels, denoted in blue; are the horizontal display width, the horizontal sync start, the horizontal sync end, and the horizontal total pixels, respectively. The vertical pixels (measured in lines), denoted in green; are the vertical display height, the vertical sync start, the end of vertical sync end, and the horizontal total pixels, respectively. The sync pulse polarities, denoted in yellow; indicate whether a given sync pulse is active low or active high. A minus symbol indicates active low and a plus symbol indicates active high. The terminology for these modeline parameters comes from The X Window System~\cite{TheOpenGroup2020}, a commonly utilized windowing system in the Linux family of operating systems where the parameters, while equivalent, are specified in a different format from the VESA standards. Equations~\eqref{eq:h_a_solve}~and~\eqref{eq:v_a_solve} show the relationship between the X Window System parameters and the VESA parameters.

    \begin{table}
        \small
        \setlength\tabcolsep{2pt}
        \begin{tabular}{| c c c c c |}
            \hline
                \textbf{\footnotesize Name} & \begin{tabular}{c} \textbf{\footnotesize Pixel} \\ \textbf{\footnotesize Clock} \\ \textbf{\footnotesize (MHz)} \end{tabular}
                & \begin{tabular}{c} \textbf{\footnotesize Horizontal} \\ \textbf{\footnotesize Parameters} \\ \textbf{\footnotesize (pixels)} \end{tabular}
                & \begin{tabular}{c} \textbf{\footnotesize Vertical} \\ \textbf{\footnotesize Parameters} \\ \textbf{\footnotesize (lines)} \end{tabular}
                & \textbf{\footnotesize Polarity} \\ \hline
                & $f_p$ & $h_D \quad h_{SS} \quad h_{SE} \quad h_{T}$ & $v_D \quad v_{SS} \quad v_{SE} \quad v_{T}$ &
                \\
                \textbf{``1920x1080\_30.00"} & {\color{red}79.75} & {\color{blue} 1920 1976 2168 2416} &  {\color{darkgreen}1080 1083 1088 1102} & {\color{olive}-hsync +vsync} \\
            \hline
        \end{tabular}
        \caption{Bandwidth requirements of a conventional display protocol}
        \label{tbl:modeline_example}
        %\end{small}
    \end{table}

    \begin{equation}
        \begin{array}{ l l l l }
            \displaystyle
            h_a=h_D & h_{fp}=h_{SS}-h_a & h_{sp}=h_{SE}-h_{SS} & h_{bp}=h_T-h_{SE} \\
            h_a=1920 & h_{fp}=1976-1920 & h_{sp}=2168-1976 & h_{bp}=2416-2168 \\
            h_a=1920 & h_{fp}=56 & h_{sp}=192 & h_{bp}=248
        \end{array}
        \label{eq:h_a_solve}
    \end{equation}

    \begin{equation}
        \begin{array}{ l l l l }
            \displaystyle
            v_a=v_D & v_{fp}=v_{SS}-v_a & v_{sp}=v_{SE}-v_{SS} & v_{bp}=v_T-v_{SE} \\
            v_a=1080 & v_{fp}=1083-1080 & v_{sp}=1088-1083 & v_{bp}=1102-1088 \\
            v_a=1080 & v_{fp}=3 & v_{sp}=5 & v_{bp}=14
        \end{array}
        \label{eq:v_a_solve}
    \end{equation}

    If the parameters for the modeline in Table~\ref{tbl:modeline_example} are placed into the formulas shown in Equations~\eqref{eq:l_h} through \eqref{eq:f_t}, the results shown in Equations~\eqref{eq:l_h_solve} through \eqref{eq:f_t_solve} are yielded. The astute reader will note that $l_h$ and $l_v$ are the same as the total width and height for the given modeline. The pixel period is $\sim12.53 ns$, meaning that each pixel is drawn for the given amount of time. The frame period is $\sim33.38 ms$, meaning that each frame is drawn for that given amount of time.

    \begin{equation}
        \begin{array}{ l }
            \displaystyle l_h=h_T=h_a+h_{fp}+h_{sp}+h_{bp} \\
            \displaystyle l_h=h_T=1920+56+192+248 \\
            \displaystyle l_h=h_T=2416
            \label{eq:l_h_solve}
        \end{array}
    \end{equation}

    \begin{equation}
        \begin{array}{ l }
            \displaystyle l_v=v_T=v_a+v_{fp}+v_{sp}+v_{bp} \\
            \displaystyle l_v=v_T=1080+3+5+14 \\
            \displaystyle l_v=v_T=1102 \\[11pt]
        \end{array}
        \label{eq:l_v_solve}
    \end{equation}

    \begin{equation}
        \begin{array}{ l }
            \displaystyle f_f={\frac{f_p}{l_h \cdot l_v}} \\[11pt]
            \displaystyle f_f={\frac{79.75e^6}{2416 \cdot 1102}} \\[11pt]
            \displaystyle f_f={29.95} \\[11pt]
        \end{array}
        \label{eq:f_f_solve}
    \end{equation}

    \begin{equation}
        p_t=12.53ns={\frac{1}{f_p}}
        \label{eq:p_t_solve}
    \end{equation}

    \begin{equation}
        f_t=33.38ms={\frac{1}{f_f}}
        \label{eq:f_t_solve}
    \end{equation}

\section{Display Protocols within IRSP Technology}
    \label{sec:displays_within_proj_system}
    IRSP technology typically utilizes conventional display protocol technology to drive IR-arrays. In the most basic form a scene generator will provide imagery that is encoded utilizing a display protocol and send it to some form of close support electronics which will then decode the stream pixel by pixel and drive an array as discussed in Chapter~\ref{chap:array_write_process}.

    For scenarios that involve unsynchronized operation where dropped frames are not an issue, these protocols can largely be used without modification. However, scenarios that require synchronization in either open loop or closed loop setups present a challenge. Often, non-standard modifications must be used to compensate for jitter among different system processes and overall system latency. Figure~\ref{fig:custom_sync} shows where in a system setup a custom solution would need to be inserted in the case of a scene generator connected directly to a CSE. In this diagram, the scene generation, NUC process, and the camera would need to be end-to-end synchronized with built-in compensation for frame latencies.

    \begin{figure}
        \centering
        \includegraphics[width=1.0\textwidth]{fig/custom_sync.pdf}
        \caption{Custom Synchronization Solution}
        \label{fig:custom_sync}
    \end{figure}

    This can range from utilizing off the shelf components such as Nvidia Quadro Sync cards~\cite{NVIDIA2020_2} or developing additional hardware pipelines capable of buffering and delaying emission of frame data such as an intermediate buffer card. This presents a particular challenge because the user typically does not have direct control over frame buffers, frame emission, or the software drivers within a system when utilizing display protocol-based technology. Moreover, encoders and decoders expect the protocols to work in a defined way that modifications for enabling synchronization could run afoul of; thus, resulting in the need for non-standard encoder and decoder implementations.

    PDP on the other hand, eases synchronization due to its nature of being packetized, which allows for controlled data to be sent when directed without the need for a custom sync solution. With a PDP based solution, a scene generator could simply synchronize to a camera (if necessary) and send frame data when required under its own direct control. Moreover, hardware links could be replaced in the future with faster technology without the need to design a new hardware specific solution for synchronization. The only requirement being that the protocol be utilized across the new link. Chapter~\ref{chap:pdp_protocol} will discuss the details of the protocol itself.

    Now that the background of display protocols has been discussed, Chapter~\ref{chap:pdp_protocol} shifts focus to a discussion of the design of PDP itself followed by a protocol specification. Some of the more protocol specific details left out here, such as variable refresh rate (VRR), are discussed there in order to compare PDP features to conventional display protocols.
