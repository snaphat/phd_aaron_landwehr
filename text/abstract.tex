Current fixed frame-rate display technology, such as, DVI, HDMI, and DisplayPort is commonly utilized for high-speed IR display systems. This technology, designed for relatively low-speed operation, incorporates a number of design decisions that limit the ability for it to meet the increasing requirements of larger resolutions and faster framerates needed within IR display systems. Firstly, it requires custom designed synchronization solutions and hardware when utilized within environments where multiple components need to be synchronized. This is because it is not designed to handle system level synchronization. Secondly, the fixed frame rate nature of the technology imposes a static requirement on frame rate across all displayed frames. This unnecessarily increases bandwidth demands by requiring the same amount of data be sent for all frames regardless of what data changes. As a result, maximum frame rate unnecessarily becomes a function of limited hardware bandwidth and image resolution.

This dissertation introduces a generalizable, dynamic, and scalable packetized display protocol (PDP) architecture. It incorporates dynamic frame rates, and high-speed capabilities to bridge the performance gaps within existing display solutions for current IR display systems. This PDP architecture eschews with many assumptions found in traditional display protocol technology. In doing so, it provides scalability, reduces bandwidth requirements, increases performance, eases synchronization burden, as well as, provides a desirable set of features for current and future IRLED Scene Projection systems. These features include dynamic sub-window (intra-frame) refresh rates, dynamic bandwidth utilization, and dynamic inter-frame refresh rates. Furthermore, this dissertation contributes a protocol specification and implementation on real hardware, coupled with, a demonstration of the benefits of this type of technology for use within high-speed IR display systems.
