%FIXME: remove "we"
\label{chap:conclusion}
This dissertation presented a Packetized Display Protocol (PDP) architecture for Infrared Scene Projection Systems (IRSPs) with the goals of providing a scalable display system that was both distributable and hardware agnostic, capable of being implemented without unnecessary complexity, capable of providing dynamic intra-frame variable refresh rates, and provide backwards compatible support for older display technology.

Firstly, the dissertation discussed the history of IRLED scene projectors and the projection process. Secondly, it provided detailed hardware and software limitations within the current technology used to drive arrays. Thirdly, it provided a problem solution to address the issues with current technology through the use of a  physical layer agnostic packetized display protocol. Fourthly, it delved into the details of using a projector system to drive arrays from end-to-end including communication between components, the interleaved array write process, and the data ordering. Fifthly, it provided an explanation of the internals of display protocols diving into how they operate at a base level; and the challenges associated with utilizing them for high speed operation. Sixthly, it provided a specification for a packetized display protocol for use with high-speed arrays. Seventhly, it provided an abstract machine model to map the protocol to current and future IRSP hardware. Eighthly, it provided a robust implementation of the protocol on real hardware. And Ninthly, it demonstrated that the use of a packetized display protocol can provide substantial benefits for IRSPs.

The PDP architecture detailed in this work has been demonstrated operating at rates of up to \mbox{2 Kilohertz} while conventual technology on the very same system hardware struggles to produce imagery at rates of only \mbox{400 hertz}. Furthermore, it has demonstrated the ability to provide fine-grained control over frame transmission and the ability to dynamically control subregion frame rates ranging from \mbox{2 hertz} to \mbox{800 hertz} while conventional technology struggles to provide support for much simpler per-frame variable refresh rates.

In point of fact, the high-speed frame rates that are detailed within this dissertation have never been achieved on high-speed IRSP technology before. And to the best of my knowledge, no other attempt to packetize display technology for high-speed multi-frame rate operation has been done before. Indeed, in researching this topic, I found very little related research topics on the subject.

At the start of this work, I set out with the intention of finding a potential solution to bridge the performance gap within current IRSP display systems. As I conclude this work, I fully believe that this goal has been achieved. And furthermore, I believe, this work provides a solid foundation for furthering research into display protocol technology.

This work focused primarily on the protocol level details of IRSPs, but there remains much research that can be done to improve the front-end side of these systems. A dynamic compositor layer could ease transition and serve as a general solution to packetizing display data dynamically. Parallel scene generators need also be developed if the field is to continue to improve performance.

Many additionally optimizations to the packetized display protocol itself and associated implementation can be done to improve frame rates further and lower overhead. Newer high-speed transport layers could be utilized to lower latency and remove unnecessary porch related bandwidth usage altogether.

Synchronization of protocol within a parallel system needs to be investigated further. For high speed operation, it could be utilized with multiple parallel drivers to double or quadruple performance with the right hardware. While not explored within this work, improvements to the analog bandwidth of a system could yield further improvements in PDP performance by allowing larger high-speed imagery to be used.

Outside the field of IRSPs, it is entirely feasible that multi-frame rate technology could be used to drive future monitors, video game systems, and virtual reality systems to achieve higher frame rates. In energy efficient systems, often data movement costs power. This technology could decrease data movement resulting in better power efficiency. Furthermore, in mobile devices, keeping the screen powered on requires high amounts of energy. This type of technology may be able to reduce power consumption requirements by driving pixels less often resulting in less overall battery load.

As for my own future work, I plan to continue to support and improve upon this technology for the foreseeable future with the hope that it will one day see standard adoption within the field of IRSPs. A Focus on dynamic frame segmentation will be one of my avenues of future research. Another avenue will be in investigating how this technology can be used to characterize array behavior and ease the non-uniformity detection process.
