%FIXME: remove "we"
\label{chap:conclusion}
In this paper, we described a packetized display protocol architecture and associated abstract machine model to convey the limitations in current fixed frame technology. Additionally, we provide an alternative display architecture that eschews the design decisions of current technology in order to provide intelligent dynamic bandwidth utilization, fine-grained control over frame transmission and synchronizationas well as allows for dynamically changing intra-frame rates. We believe this architecture has the potential to provide the capabilities to bridge the performance gap found in current technology, and will serve as a better-fit solution for future high performance IRLED systems due to the scalable nature of the design and the carefully incorporated abstraction tailored to allow for different types of hardware and system setups to utilize the PDP architecture. Care has been taken in the design to incorporate many different possible system setups without limiting the use case of PDP to a specific hardware setup; while at the same time, considering firmware implementation and timing aspects to packet decoding.

Current work includes a FPGA based implementation of a PDP decoder architecture utilizing HDMI. We have provided both a description of the implemented architectureas well as simulated sample data running on the architecture. Future work includes testing the architecture on an emitter array, performing scalability testing, and comparing the results to a conventional architecture at matching pixel clock rates in order to show effective speedup with varying packet sizes. Further work is to be done to demonstrate dynamic frame rates in action on an array. Finally, a CRC is to be implemented to ensure correct operation at all times. We also wish to scale the number of inputs in order to increase the effective hardware bandwidth further than capable with a conventional system.
