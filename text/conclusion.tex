%FIXME: remove "we"
\label{chap:conclusion}
This dissertation presented a Packetized Display Protocol (PDP) architecture for Infrared Scene Projection Systems (IRSPs) with the goals of providing a scalable display system that was both distributable and hardware agnostic, capable of being implemented without unnecessary complexity, capable of providing dynamic intra-frame variable refresh rates, and provide backwards compatible support for older display technology.

Firstly, the dissertation discussed the history of IRLED scene projectors and the projection process. Secondly, it provided detailed hardware and software limitations within the current technology used to drive arrays. Thirdly, it provided a problem solution to address the issues with current technology through the use of a  physical layer agnostic packetized display protocol. Fourthly, it delved into the details of using a projector system to drive arrays from end-to-end including communication between components, the interleaved array write process, and the data ordering. Fifthly, it provided an explanation of the internals of display protocols diving into how they operate at a base level; and the challenges associated with utilizing them for high speed operation. Sixthly, it provided a specification for a packetized display protocol for use with high-speed arrays. Seventhly, it provided an abstract machine model to map the protocol to current and future IRSP hardware. Eighthly, it provided a robust implementation of the protocol on real hardware. And Ninthly, it demonstrated that the use of a packetized display protocol can provide substantial benefits for IRSPs.

The PDP architecture detailed in this work has been demonstrated operating at rates of up to \mbox{2 Kilohertz} while conventual technology on the very same system hardware struggles to produce imagery at rates of only \mbox{400 hertz}. Furthermore, it has demonstrated the ability to provide fine-grained control over frame transmission and the ability to dynamically control subregion frame rates ranging from \mbox{2 hertz} to \mbox{800 hertz} while conventional technology struggles to provide support for much simpler per-frame variable refresh rates.

At the start of this work, I set out with the intention of finding a potential solution to bridge the performance gap within current IRSP display systems. As I conclude this work, I fully believe that this goal has been achieved. And furthermore, I believe, this work provides a solid foundation for furthering research into display protocol technology.

This work focused primarily on the protocol level details of IRSPs, but there remains much research that can be done to improve the front-end side of these systems. A dynamic compositor layer could ease transition and serve as a general solution to packetizing display data dynamically. Parallel scene generators need also be developed if the field is to continue to improve performance.

Many additionally optimizations to the packetized display protocol itself and associated implementation can be done to improve frame rates further and lower overhead. Newer high-speed transport layers could be utilized to lower latency and remove unnecessary porch related bandwidth usage altogether.

Synchronization of protocol within a parallel system needs to be investigated further. For high speed operation, it could be utilized with multiple parallel drivers to double or quadruple performance with the right hardware. While not explored within this work, improvements to the analog bandwidth of a system could yield further improvements in PDP performance by allowing larger high-speed imagery to be used.

Outside the field of IRSPs, it is entirely feasible that a technology similar to this could be used to drive future monitors, video game systems, and virtual reality systems. These avenues could be explored to see if there is place for packetized display protocols within those applications.

As for my own future work, I plan to continue to support and improve upon this technology for the foreseeable future with the hope that it will one day see standard adoption within the field of IRSPs. A Focus on dynamic frame segmentation will be one of my avenues of future research. Another avenue will be in investigating how this technology can be used to characterize array behavior and ease the non-uniformity detection process.



%In this paper, we described a packetized display protocol architecture and associated abstract machine model to convey the limitations in current fixed frame technology. Additionally, we provide an alternative display architecture that eschews with the design decisions of current technology to provide intelligent dynamic bandwidth utilization, fine-grained control over frame transmission and synchronization as well as allows for dynamically changing intra-frame rates.

%We believe this architecture has the potential to provide the capabilities to bridge the performance gap found in current technology, and serves as a better-fit solution for future high performance IRSP systems due to the scalable nature of the design and the carefully incorporated abstraction tailored to allow for different types of hardware and system setups to utilize the PDP architecture. Care has been taken in the design to incorporate many different possible system setups without limiting the use case of the PDP to a specific hardware setup; while at the same time, considering firmware implementation and timing aspects to packet decoding.

%Current work includes a FPGA based implementation of a PDP decoder architecture utilizing HDMI. We have provided a description of the implemented architecture as well as simulated sample data running on the architecture. Future work includes testing the architecture on an emitter array, performing scalability testing, and comparing the results to a conventional architecture at matching pixel clock rates to show effective speedup with varying packet sizes. Further work is to be done to demonstrate dynamic frame rates in action on an array. Finally, a CRC is to be implemented to ensure correct operation at all times. We also wish to scale the number of inputs to increase the effective hardware bandwidth further than capable with a conventional system.
